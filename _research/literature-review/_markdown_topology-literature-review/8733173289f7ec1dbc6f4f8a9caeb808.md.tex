\markdownRendererDocumentBegin
\markdownRendererStrongEmphasis{tldr;} Comprehensive review of the field of Artificial Chemistry, the ways to define artificial chemistry systems, different characteristics and methods, various approaches, considerations when modeling living systems and key questions related to modeling specific systems.\markdownRendererInterblockSeparator
{}\markdownRendererUlBegin
\markdownRendererUlItem working hypothesis of Artificial Life research: biology can be modeled using complex systems of many interacting components, as there’s emergence of novel global behavior from local interactions that’s not predictable from the properties of its components but is rather due to their organization/function\markdownRendererUlItemEnd 
\markdownRendererUlItem Artificial Chemistry (AC) should be able to explain the origin of evolutionary systems leading to the evolution of life\markdownRendererUlItemEnd 
\markdownRendererUlItem AC has three main dimensions in its application: modeling (of biological/evolutionary/social systems), information processing (investigating chemical systems that perform computation), optimization (using chemical systems to find solutions to combinatorial problems)\markdownRendererUlItemEnd 
\markdownRendererUlItem we define an AC as (S, R, A): S being the set of molecules, R the collision rules, A the algorithm (i.e., vessel where the reactions take place)\markdownRendererUlItemEnd 
\markdownRendererUlItem set of molecules S: can be abstract symbols, character sequences, lambda-expressions, binary strings, numbers, etc.\markdownRendererUlItemEnd 
\markdownRendererUlItem set of rules R: chemical notation describing the reactions\markdownRendererUlItemEnd 
\markdownRendererUlItem algorithm A: simplest case is a multiset or concentration vector—can be modeled as an algorithm that draws molecules randomly, using PDEs, etc.\markdownRendererUlItemEnd 
\markdownRendererUlItem alternative definition using the (S, I) notation: S the set and I the interactions, can be preferable if on a lattice\markdownRendererUlItemEnd 
\markdownRendererUlItem characteristics and methods of an AC system include: the definition of molecules (explicit or implicit), reaction laws (idem), dynamics (idem), level of abstraction (analogous or abstract), constructive dynamics (if the dynamics can change over time), random chemistries (systems based on random rules/rate constants), time scale, pattern matching, spatial topology\markdownRendererUlItemEnd 
\markdownRendererUlItem approaches: see Table~\ref{tab:dittrich2001}\markdownRendererUlItemEnd 
\markdownRendererUlItem we can use AC systems to study living organisms as information processing units\markdownRendererUlItemEnd 
\markdownRendererUlItem several phenomena are commonly observed in AC systems: reduction of diversity, formation of densely coupled stable networks, syntactic and semantic closure, evolution and punctuated equilibrium\markdownRendererUlItemEnd 
\markdownRendererUlItem important questions include: what level of abstraction is appropriate? which key ingredients? how much physical/chemical knowledge to include? et cetera.\markdownRendererUlItemEnd 
\markdownRendererUlEnd \markdownRendererDocumentEnd