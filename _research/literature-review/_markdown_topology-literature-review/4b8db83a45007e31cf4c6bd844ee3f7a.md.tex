\markdownRendererDocumentBegin
\markdownRendererStrongEmphasis{Summary:} Serpentinization of ultramafic rock exhibit nanoporosity that may—combined with a fractal network of pores—provide energy sources for the deep biosphere, change the physical properties of confined fluids and influence drastically the production of abiotic organics\markdownRendererInterblockSeparator
{}\markdownRendererUlBegin
\markdownRendererUlItem serpentinization of ultramafic rock exhibit nanoporosity that impact the geochemical behaviour ; can provide energy sources (e.g. H and CH4) for the deep biosphere\markdownRendererUlItemEnd 
\markdownRendererUlItem here we investigate the characteristics, scale dependence and implications of nanoporosity\markdownRendererUlItemEnd 
\markdownRendererUlItem using multidimensional imaging and molecular dynamics we conclude that serpentinites function as nanoporous media with pores < 100nm\markdownRendererUlItemEnd 
\markdownRendererUlItem the fractale nature of pore size distribution support the presence of a nanoporous network with induces emergent properties in fluid transport, mineral solubility and chemical reactions\markdownRendererUlItemEnd 
\markdownRendererUlItem while tectonic deformation and thermal cracking can generate fluid pathways in rocks, the mineral reaction itself can also induce differential stress causing the rock to fracture\markdownRendererUlItemEnd 
\markdownRendererUlItem for this study samples representing the oceanic lithospheric mantle and exhumed mantle on land were collected through the Ocean Drilling Program at the Mid-Atlantic Ridge and in an ultramafic complex in Norway—both samples are composed of lizardite\markdownRendererUlItemEnd 
\markdownRendererUlItem there is currently an absence of a robust model that elucidates the genesis of nanoporosity in serpentinites\markdownRendererUlItemEnd 
\markdownRendererUlItem analysis reveals that serpentine pores are generally elongated rather than circular—which may be attributed to the underlying mineral formation process\markdownRendererUlItemEnd 
\markdownRendererUlItem moreover, pore size distribution exhibits fractal characteristics—suggesting that pores at various scales can be attributed to the same underlying processes\markdownRendererUlItemEnd 
\markdownRendererUlItem serpentinites function as nanoporous media with pores of less than 100nm—fluid restricted to these scales exhibit distinct behaviours which could influence fluid and mass transport or change the physical properties of the fluid—molecular dynamics simulations suggest that the reduction of CO2 to CH4 is favoured in these conditions, possibly having drastic effects on e.g. the production of abiotic organics\markdownRendererUlItemEnd 
\markdownRendererUlEnd \markdownRendererDocumentEnd