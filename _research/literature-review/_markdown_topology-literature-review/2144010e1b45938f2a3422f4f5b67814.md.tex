\markdownRendererDocumentBegin
\markdownRendererStrongEmphasis{tldr;} The authors present a model of artificial chemistry that combine hydration-dehydration cycles with sequence replication on a diffusive lattice, and demonstrate that limited diffusion can provide an alternative to compartmentalization.\markdownRendererInterblockSeparator
{}\markdownRendererUlBegin
\markdownRendererUlItem the authors present a model for what was an earlier stage of evolution than e.g. refinement of preexisting enzymes (like replicase)—the model includes environmental cycles (e.g. hydration-dehydration) which coincide with different monomer/polymer diffusivity—it is also observed that polymers form cluster/aggregates\markdownRendererUlItemEnd 
\markdownRendererUlItem we introduce the term Universal Sequence Replication (USR) to represent the possibility that prebiotic template-directed synthesis provided means for the replication of polymers, regardless of monomer sequence (i.e. replicative rate constants are sequence-independent)\markdownRendererUlItemEnd 
\markdownRendererUlItem Kinetic Monte Carlo simulations are used to evolve populations of informational polymers, formed by spontaneous polymerization, replicated by USR and then subject to hydrolysis in a diffusion-limited environment—resulting in species diversity remaining high despite functional selection not dominating the system\markdownRendererUlItemEnd 
\markdownRendererUlItem cycling: polymerization via spontaneous assembly and USR via template-directed synthesis occur in the hot-dry conditions whereas polymer degradation and diffusion of monomers and polymers occur in the cool-wet conditions—functional polymers only exhibit catalytic activity during the hydrated phase that promotes the folding into the active state\markdownRendererUlItemEnd 
\markdownRendererUlItem reversible polymerization: polymers degrading into monomers are added back into the population, creating feedback between both concentrations\markdownRendererUlItemEnd 
\markdownRendererUlItem monomer concentrations: the total number of monomeric units (e.g. nucleotides) is constant, with species labeled A and B\markdownRendererUlItemEnd 
\markdownRendererUlItem surface confinement and limited diffusion: polymers and monomers are confined to a 2D surface, where diffusion takes place in the hydrated phase—size 64x64\markdownRendererUlItemEnd 
\markdownRendererUlItem USR: all polymers have the same rate constant for replication\markdownRendererUlItemEnd 
\markdownRendererUlItem emergence of a functional polymer: we explore the emergence of the first functional informational polymer, here referred to as an A-zyme, capable of synthesizing A monomers from proto-A\markdownRendererUlItemEnd 
\markdownRendererUlItem polymer species: we choose a fixed length of RL=20\markdownRendererUlItemEnd 
\markdownRendererUlItem spatial patterning: we observe spatial patterning of polymers that is dependent on the diffusivities, and which can be interpreted as an indirect form of cooperativity between high density populations (fig3)\markdownRendererUlItemEnd 
\markdownRendererUlItem emergence of the Azyme yields a selective advantage (see text and fig5)\markdownRendererUlItemEnd 
\markdownRendererUlItem conclusion: limited movement (i.e., diffusion) could have been shown to provide a possible alternative to encapsulation (see fig4)—put differently, that physical compartmentalization is not necessary for prebiotic evolution\markdownRendererUlItemEnd 
\markdownRendererUlEnd \markdownRendererDocumentEnd