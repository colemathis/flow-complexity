\markdownRendererDocumentBegin
\markdownRendererStrongEmphasis{tldr}; Using lambda-calculus to define an artificial chemistry system, the authors suggest that three invariants of evolution would inevitably reappear “if the tape were played twice”: hypercycles, self-sustaining organizations, more complex metaorganizations\markdownRendererInterblockSeparator
{}\markdownRendererUlBegin
\markdownRendererUlItem the authors develop an abstract chemistry implemented on lambda-calculus, and argue that the following features (that emerge in their model) arise generically: hypercycles, self-maintaining organizations, high-order self-maintaining organizations\markdownRendererUlItemEnd 
\markdownRendererUlItem Gould: would the biodiversity that surrounds us be different if “the tape were played twice”, i.e. contingency vs necessity\markdownRendererUlItemEnd 
\markdownRendererUlItem we can investigate this questions using a model—this model cannot assume the prior existence of organisms\markdownRendererUlItemEnd 
\markdownRendererUlItem chemistry is based on the syntactic structure in which the molecules are assembled\markdownRendererUlItemEnd 
\markdownRendererUlItem in lambda-calculus, objects are defined inductively in terms of a nonlinear combination of primitives\markdownRendererUlItemEnd 
\markdownRendererUlItem this implies that each object is a function, such that we can write (A)B, or “the action of A on B”, which implies the “reduction in normal form” of (A)B --> C1 --> C2, …or, equivalently, (A)B=C\markdownRendererUlItemEnd 
\markdownRendererUlItem lambda-calculus isn’t a theory of actual chemistry, e.g. no reference to thermodynamics—however TD driving is abstracted by requiring that every object is in normal form—or to spatial constraints, conservation laws, etc.\markdownRendererUlItemEnd 
\markdownRendererUlItem the basic event in lambda-chemistry is a collision between two objects, happening inside a well-stirred flow reactor whose number of objects is kept constant\markdownRendererUlItemEnd 
\markdownRendererUlItem level 0 experiments: 1000 random functions, one copy each—these experiments always become dominated by self-copying functions or hypercyclically coupled copying functions a la Eigen\markdownRendererUlItemEnd 
\markdownRendererUlItem level 1 experiments: restricting copying functions generates complexity (see text and fig1) whose organizations are self-maintaining (instead of self-reproducing) and are very robust to perturbations (e.g., with random objects)\markdownRendererUlItemEnd 
\markdownRendererUlItem level 2 experiments: initiated with the products of two level 1 experiments and have two outcomes, either a single level 1 organization dominates or a metaorganization arises with metabolic flows\markdownRendererUlItemEnd 
\markdownRendererUlItem the results of these experiments can be summarized by three claims: hypercycles arise, when replication is prohibited complex self-maintaining organizations emerge, organizations can be hierarchically combined to produce new organizations\markdownRendererUlItemEnd 
\markdownRendererUlItem the biological analogy to what precedes would be: the emergence of self-replication, self-maintaining prokaryotic organizations, self-maintaining eukaryotic organizations\markdownRendererUlItemEnd 
\markdownRendererUlItem this challenges the view that Darwinian selection presupposes the existence of self-reproducing entities as these are absent from level 1-2\markdownRendererUlItemEnd 
\markdownRendererUlItem this study also challenges the hypothetical, simultaneous emergence of self-reproduction and self-maintenance\markdownRendererUlItemEnd 
\markdownRendererUlEnd \markdownRendererDocumentEnd