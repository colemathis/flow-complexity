\markdownRendererDocumentBegin
\markdownRendererStrongEmphasis{Summary:} Pathway Complexity is a (graph-based) method to measure the complexity of objects by calculating the number of steps required for their construction. It could be used as an agnostic biosignature, and to inform a new theory of biology.\markdownRendererInterblockSeparator
{}\markdownRendererUlBegin
\markdownRendererUlItem One thing that discriminates living systems is their ability to generate non-random structures in a large abundance—folded protein, lilving cells, microbial communities\markdownRendererUlItemEnd 
\markdownRendererUlItem There’s been many proposals for finding biosignatures: gases (e.g., methane), Fe isotope ratio, biological impact on minerals, distribution in monomer abundance, red edge, etc.\markdownRendererUlItemEnd 
\markdownRendererUlItem Two difficulties with biosignatures: casting our net wide enough (i.e., not remaining tied to terran biology) and avoiding false positives\markdownRendererUlItemEnd 
\markdownRendererUlItem Prior definitions of complexity: Shannon entropy, Kolmogorov complexity, logical depth, effective complexity, computational complexity, stochastic complexity—but they sometimes give misleading results (e.g., maximal values for random structures)\markdownRendererUlItemEnd 
\markdownRendererUlItem Pathway Complexity is a measure of complexity based on the construction of an object through joining operations of basic substructures, where the complexity is defined as the number of associated joining operations\markdownRendererUlItemEnd 
\markdownRendererUlItem The reasoning behind Pathway Complexity is that objects of sufficient complexity would have their formation compete against the combinatorial explosion of possible objects—i.e., producing non-trivial construction trajectories would be a characteristic unique to life, which also makes the approach agnostic\markdownRendererUlItemEnd 
\markdownRendererUlItem We can use a graph-based approach to evaluate the construction of objects\markdownRendererUlItemEnd 
\markdownRendererUlItem There are both lower- and upper-bounds on the complexity of objects, and living systems produce objects that fall between these two extremes\markdownRendererUlItemEnd 
\markdownRendererUlItem We can also use a variant called “recursive tree complexity” where we partition the object into subgraphs\markdownRendererUlItemEnd 
\markdownRendererUlItem Pathway Complexity can be useful in the lab in exploring the threshold between living/non-living, and could inform a new theory for biology\markdownRendererUlItemEnd 
\markdownRendererUlEnd \markdownRendererDocumentEnd