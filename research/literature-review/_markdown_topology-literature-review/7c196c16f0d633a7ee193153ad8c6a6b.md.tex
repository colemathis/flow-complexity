\markdownRendererDocumentBegin
\markdownRendererStrongEmphasis{Chapter 1: what is life?}\markdownRendererInterblockSeparator
{}\markdownRendererUlBeginTight
\markdownRendererUlItem 7: does “life” exist? is it a property of the parts? can it be explainable by the known laws of physics and chemistry?\markdownRendererUlItemEnd 
\markdownRendererUlItem 11: there is no “magic transition point” where a (collection of) molecule is suddenly living\markdownRendererUlItemEnd 
\markdownRendererUlItem 26: instead of answering “what is life?” we might have to transition to answering “why life?” and similarly go from defining life to deriving life’s properties from a (new) fundamental physical theory\markdownRendererUlItemEnd 
\markdownRendererUlItem 29: biologists are too focused on defining life in terms of features observed on Earth, astrobiologists are too anthropocentric too, chemists think it’s all chemistry, computer scientists only focus on the software, physicists focus on thermodynamics, etc.\markdownRendererUlItemEnd 
\markdownRendererUlEndTight \markdownRendererInterblockSeparator
{}\markdownRendererStrongEmphasis{Chapter 2: hard problems}\markdownRendererInterblockSeparator
{}\markdownRendererUlBeginTight
\markdownRendererUlItem 32: three fundamental problems for science: origins of matter, life, mind\markdownRendererUlItemEnd 
\markdownRendererUlItem 40: in order to test a theory, it must have measurable consequences on the world\markdownRendererUlItemEnd 
\markdownRendererUlItem 44: focusing on chemical correlates (e.g.: compartmentalization) has, after decades, yielded only limited progress—are we missing something more fundamental? we need new physics: “how is it that information can cause things?”\markdownRendererUlItemEnd 
\markdownRendererUlItem 53: the three hard problems can be framed as “why do some things exist and not others?”\markdownRendererUlItemEnd 
\markdownRendererUlEndTight \markdownRendererInterblockSeparator
{}\markdownRendererStrongEmphasis{Chapter 3: life is what?}\markdownRendererInterblockSeparator
{}\markdownRendererUlBeginTight
\markdownRendererUlItem 70: we have not yet built a physics that deals with the combinatorial space of all possible objects that can be built from elementary stuff\markdownRendererUlItemEnd 
\markdownRendererUlItem 71: in order to explain these objects, we have to reconceptualize the space of combinatorial possibilities as a physical space\markdownRendererUlItemEnd 
\markdownRendererUlItem 73: we have to formalize the foundations of a theory for life not in terms of what life is, but of what life does—as life is the only thing in the universe that can make objects that are composed of many unique, recursively constructed parts\markdownRendererUlItemEnd 
\markdownRendererUlItem 74: the key hypothesis of assembly theory is that the observation of complex objects implies that the “information” about the steps of their formation must exist in other objects (implies a physically instantiated memory) which can happen only via evolution or learning—if selection acts as a constraint that can drive the emergence of life, we need to discover the laws of physics that explain selection\markdownRendererUlItemEnd 
\markdownRendererUlItem 75: current physics puts all causation at the microscale, whereas in assembly theory almost no causation exists in elementary objects—causation is built up over time along lineages\markdownRendererUlItemEnd 
\markdownRendererUlItem 76: the copy number—how many of a given object you observe—is of fundamental importance in defining a theory that accounts for selection, as the more parts it has the less likely an exact copy can exist without some precise mechanism that has itself been selected generates the object\markdownRendererUlItemEnd 
\markdownRendererUlItem 78: the assembly index is the way we formalize how hard it is for the universe to build something—defined by the smallest number of physically possible steps necessary to produce an object—the assembly space captures the minimal memory, in terms of the minimal number of operations necessary to assembly the observed object, based on objects that could have existed in its past\markdownRendererUlItemEnd 
\markdownRendererUlItem 84: the minimal path in assembly theory is agnostic and does not depend on the chemistry of life as we know it, we can go in the lab and measure it using several different instruments, and for any molecule\markdownRendererUlItemEnd 
\markdownRendererUlItem 87: distinguishing design (via an intelligent agent or an evolutionary process) from randomness is important in the context of detecting alien intelligences in the universe, via the signatures of their technology\markdownRendererUlItemEnd 
\markdownRendererUlItem 88: the idea of comploxity peak goes back to earlier work in complex systems, including that of Jim Crutchfield and colleagues demonstrating how a peak in complexity somewhere between fully ordered and totally random (high entropy) characterizes the behavior of many systems\markdownRendererUlItemEnd 
\markdownRendererUlItem 95: most of the excitement around what we are doing with assembly theory has so far focused on how it provides the first theoretically motivated and empirically tested solution to the problem of ruling out false positives in alien life detection\markdownRendererUlItemEnd 
\markdownRendererUlItem 97: we need to define a threshold assembly index and copy number that determines when an object must have been produced by a causal chain of stacked objects, meaning life\markdownRendererUlItemEnd 
\markdownRendererUlItem 97: to make an alien life-detection system, the first step was to confirm that assembly index can be measure with laboratory instrumentation—what Lee’s lab found is that the assembly index of a molecule correlates directly with features of the fragmentation pattern of the molecule\markdownRendererUlItemEnd 
\markdownRendererUlItem 99: assembly index is an intrinsic property of a molecule—it is a feature we can find via a multimodal suite of different measuring techniques (infrared spectroscopy and nuclear magnetic resonance)\markdownRendererUlItemEnd 
\markdownRendererUlItem 99: what we observed experimentally is that this threshold in chemical space for observing objects made by life (and only life) is fifteen steps (for the chemical space on Earth at least\markdownRendererUlItemEnd 
\markdownRendererUlItem 100: fifteen is about where we hit the threshold of one copy per mole without selection\markdownRendererUlItemEnd 
\markdownRendererUlItem 101: work done in Sara’s lab aims to refine the theory around this threshold as a phase transition in the possibility of objects existing (carried out by Dániel Czégel, Gage Seibert, and Swanand Khanapurk\markdownRendererUlItemEnd 
\markdownRendererUlEndTight \markdownRendererInterblockSeparator
{}\markdownRendererStrongEmphasis{Chapter 4: aliens}\markdownRendererInterblockSeparator
{}\markdownRendererUlBeginTight
\markdownRendererUlItem “NASA started investing more concretely in astrobiology research, founding shortly thereafter the NASA Astrobiology Institute (NAI). Over its twenty years in operation between 1998 and 2019, the NAI launched many of astrobiology’s most important research agendas and fostered a new generation of scientists tackling the open questions of astrobiology. In 2017, the annual NASA Space Act declared the search for alien life among the primary objectives of NASA for the first time in history. Astrobiology is now a major theme throughout several NASA mission directorates” (Walker, 2024, p. 119)\markdownRendererUlItemEnd 
\markdownRendererUlItem “A fundamental understanding of life will begin a new era of astrobiology, because it will allow us to move beyond the search for life as one of analogy. It will allow us to search for life as we don’t know it. Indeed, it will allow us to search for life as no one knows it. Understanding life in the abstract will allow us to predict what alien examples of life could be like. Our hope with assembly theory is that we will be able to use it to detect alien life, if it is out there, and that even before we make first contact we also might use it to predict features of the possibility spaces aliens might live in.” (Walker, 2024, p. 128)\markdownRendererUlItemEnd 
\markdownRendererUlItem “As my geochemistry colleagues at Arizona State University Everett Shock and Hilairy Hartnett advocate, “Biochemistry is what the Earth allows.”[5] (Walker, 2024, p. 129)\markdownRendererUlItemEnd 
\markdownRendererUlItem “consider all of the microenvironments across a planetary surface as microreactors, and a planet as an engine exploring via these microenvironments the combinatorial space of geochemically possible small molecules.[7] Fluctuations in the density of assembly happen, and life emerges as a cascade of objects building into the high assembly universe.” (Walker, 2024, p. 131)\markdownRendererUlItemEnd 
\markdownRendererUlItem “When we can run origin-of-life experiments at scale, they will allow us to predict how much variation we should expect in different geochemical environments. That is, we will not only be able to speculate that life elsewhere should be different, but we should be able to say how different with quantitative precision.” (Walker, 2024, p. 133)\markdownRendererUlItemEnd 
\markdownRendererUlItem “A very different sense of universality, found in physics, could be useful for guiding our search for alternative forms of life. That concept arises from the study of phase transitions. Familiar examples of phase transitions include when ice melts or when liquid water vaporizes. Physicists refer to the physical conditions where a phase transition occurs as a critical point. In the study of phase transitions, an important mathematical tool has been the identification of scaling laws, which allow the quantification of the systematic variation of one property with another—for example, how temperature might scale with pressure. Universality classes in physics are then identified by systems that have the same scaling behavior.[11] Remarkably, this can occur across systems that are completely different in terms of their component parts and microscopic physics: the water-gas phase transition has the same scaling behavior as the one that occurs in magnetic materials between the ordered and disordered phases (where magnets align or point in random directions respectively). Physicists have used this to predict properties of new materials that might be in the same universality class as existing ones—e.g., share the same macroscale properties (scaling) while varying in lower-level microscopic detail (whether the system is made of water or a magnetic material). This offers the possibility that scaling laws identified for life’s chemistry might enable us to predict features of alien chemistry. The first challenge is to determine if generalizable trends indeed do exist.” (Walker, 2024, p. 134)\markdownRendererUlItemEnd 
\markdownRendererUlItem “There are two resolutions to the challenges of life detection I’ve posed. Either we need biosignatures that are not subject to false positives, or we need to determine the probability for an origin-of-life event. Assembly theory solves the problem of false positives, because a falsifiable hypothesis of the theory is that life is the only mechanism that can produce high assembly objects.” (Walker, 2024, p. 148)\markdownRendererUlItemEnd 
\markdownRendererUlItem “we are currently working on getting assembly theory “flight-ready” by calculating how feasible it is to make a detection of high assembly molecules using current mass-spec instrumentation from prior and upcoming NASA missions. This could prove critically important for analyzing data from solar system worlds like Titan, which the NASA Dragonfly mission will visit in the coming decades.” (Walker, 2024, p. 148)\markdownRendererUlItemEnd 
\markdownRendererUlEndTight \markdownRendererInterblockSeparator
{}\markdownRendererStrongEmphasis{Chapter 5: origins}\markdownRendererInterblockSeparator
{}\markdownRendererUlBeginTight
\markdownRendererUlItem “That conversation marked the beginning of our moon-shot project to solve the origin of life and first contact with alien life at the same time. We are going to need a very large experiment for this one. Planets are not easy to simulate with physical material, but this is not a simulation we can run in a computer. We would not even know what to program into a simulation, because we don’t know how chemistry generates life.” (Walker, 2024, p. 153)\markdownRendererUlItemEnd 
\markdownRendererUlItem “We must build a physical experiment to simulate planetary conditions to search for the origin of alien life for a similar reason. We need to test our theories against reality, and we can do so only in experiments that can explore the high combinatorial diversity of the chemical universe.” (Walker, 2024, p. 154)\markdownRendererUlItemEnd 
\markdownRendererUlItem “How then can we design an experiment and expect to observe the spontaneous emergence of life, e.g., via the spontaneous emergence of design we did not put in? The boundaries of any experiment we make to produce an origin-of-life event will always be set by information and constraints that are themselves the product of an origin-of-life event. This problem is not widely discussed in the origin-of-life literature, but it should be. My collaborators and I are aiming to get ahead of it by identifying how to quantify our agency and the informational constraints we impose on chemistry when we do origin-of-life experiments. We need to do this if we are to solve how life happens without us. To do so, we need to make an information vacuum” (Walker, 2024, p. 157)\markdownRendererUlItemEnd 
\markdownRendererUlItem “Because the chemputer has digitized chemistry—it runs explicitly on an algorithmic procedurewe can precisely quantify how much information we are putting into the experiment. Because we have assembly theory, we can quantify the assembly of the starting molecules and track how assembled things become over time to look for evidence of the emergence of evolution and selection within the boundary conditions of the experiment. These two things when combined allow us to quantify how much of an “information vacuum” we have created via the experiment itself being a part of a lineage of life, versus how much assembly the chemistry in the experiment generates de novo.” (Walker, 2024, p. 169)\markdownRendererUlItemEnd 
\markdownRendererUlEndTight \markdownRendererInterblockSeparator
{}\markdownRendererStrongEmphasis{Chapter 6: planetary futures}\markdownRendererDocumentEnd