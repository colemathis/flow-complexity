\markdownRendererDocumentBegin
\markdownRendererStrongEmphasis{tldr;} Provides an analysis, using an artificial chemistry system, of self-sustainability and heredity, and offers a formal definition that allows to determine whether a system is self-sustaining.\markdownRendererInterblockSeparator
{}\markdownRendererUlBegin
\markdownRendererUlItem the ability to self-sustain is central to life, but a definition of self-sustainability is missing\markdownRendererUlItemEnd 
\markdownRendererUlItem connecting self-sustainability with heredity, another feature central to life, is another problem\markdownRendererUlItemEnd 
\markdownRendererUlItem here the author proposes a definition of self-sustainability taking into account the chemical reaction network and the external environment, simplified as a continuous-flow stirred tank reactor (CSTR)\markdownRendererUlItemEnd 
\markdownRendererUlItem self-sustaining can either mean: “no-intervention” (e.g., NASA definition) or “regeneration” (e.g., autopoiesis)\markdownRendererUlItemEnd 
\markdownRendererUlItem no-intervention school: RNA amplification is self-sustaining as it can be continued indefinitely\markdownRendererUlItemEnd 
\markdownRendererUlItem regeneration school: each molecule can be produced from the food source (RAF theory), if topologically all molecules consumed are also produced (chemical organization theory)—three issues with this latter definition (topological, defined wrt. molecules, too stringent)\markdownRendererUlItemEnd 
\markdownRendererUlItem origin of life: would require the origin of self-sustainability and the origin of heredity—this is why we question theories of self-sustaining chemistry without heredity\markdownRendererUlItemEnd 
\markdownRendererUlItem we define the model as a set of ODEs describing the mean field dynamics (see eq. 2)\markdownRendererUlItemEnd 
\markdownRendererUlItem two assumptions: all molecules are uniformly distributed, all species have the same molar volume --> solution behaves like an ideal gas\markdownRendererUlItemEnd 
\markdownRendererUlItem we can define mathematically self-sustainability wrt. the inflow f and the initial conditions xi\markdownRendererUlItemEnd 
\markdownRendererUlItem we can define multiple concepts related to self-sustainability: trivial, self-sustaining, impossible, sequential\markdownRendererUlItemEnd 
\markdownRendererUlItem we suggest that a self-driven CRN is irreducible self-sustaining if it satisfies the criteria for “overproduction” and “no-over-intake”\markdownRendererUlItemEnd 
\markdownRendererUlItem moreover, if a CRS is self-sustaining, it has preliminary heredity: self-sustainability’s trigger molecule can by considered a “holistic, limited hereditary replicator”\markdownRendererUlItemEnd 
\markdownRendererUlItem whether a system is self-sustaining also depends on the environment\markdownRendererUlItemEnd 
\markdownRendererUlEnd \markdownRendererDocumentEnd